\begin{resumo}
A mineração de dados tem se tornado um importante fator estratégico dentro do meio empresarial, uma vez que permite que informações não triviais sejam identificadas de forma a alavancar os negócios ou mesmo promover inovações de produtos e serviços. Neste contexto, a classificação de dados é uma das mais importantes e recorrentes tópicos encontrades na mineração de dados e dessa maneira o seguinte trabalho tem por objetivo propor um algoritmo multiobjetivo baseado em PSO para classificação de dados por meio de extração de regras. Nesta abordagem, cada partícula (solução candidata) representa uma regra de classificação que é convertida em um predicado lógico para subsequente utilização em uma operação de seleção na linguagem SQL para avaliação de desempenho do algoritmo. Em experimentos realizados revelam que a abordagem proposta se mostrou competitiva, com resultados promissores quando comparados à outros métodos de classificação clássicos reconhecidos na literatura e apresentar resultados satisfatório especialmente em bases de dados desabalanceadas. 

\vspace{1.5ex}

\noindent \textbf{Palavras-chave}: Otimização por Enxame de Partículas, Inteligência de Enxames, Abordagem Multiobjetivo, Extração de Regras e Classificação de Dados. 
\end{resumo}
