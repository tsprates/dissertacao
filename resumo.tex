\begin{resumo}
A mineração de dados tem se tornado um importante fator estratégico dentro do meio empresarial, uma vez que permite que informações não triviais sejam identificadas de forma a alavancar os negócios ou mesmo promover inovações de produtos e serviços. Neste contexto, a classificação de dados é reconhecida como uma das mais importantes, e recorrentes, tópicos encontrados na mineração de dados. Dessa maneira, o seguinte trabalho tem por objetivo propor um algoritmo multiobjetivo baseado em PSO para classificação de dados por meio de extração de regras. Nesta abordagem, cada partícula (solução candidata), representada uma regra de classificação, é convertida em um predicado lógico para construção de uma operação de seleção na linguagem SQL para avaliação de desempenho do algoritmo. Durante a realização de experimentos computacionais, foi observado que a abordagem proposta se mostrou competitiva, com resultados promissores quando comparados à outros métodos de classificação clássicos, reconhecidos na literatura, destacando-se especialmente em bases de dados desabalanceadas. 

\vspace{1.5ex}

\noindent \textbf{Palavras-chave}: Otimização por Enxame de Partículas, Inteligência de Enxames, Abordagem Multiobjetivo, Extração de Regras e Classificação de Dados. 
\end{resumo}
